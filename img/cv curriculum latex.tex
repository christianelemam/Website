% Righe di impostazioni per TeXworks e TeXstudio

% !TEX encoding = UTF-8

% !TEX program = pdflatex

% !TEX spellcheck = en_US


\documentclass[11pt,a4paper,sans]{moderncv}

%lo stile del nostro curriculum

% come vedete ho scelto classic ma potete scegliere

% 'moderncv' 'casual' (si default),'classic', 'oldstyle' o 'banking'

% e i colori 'blue', 'orange', 'green', 'red', 'purple', 'grey' o 'black'

\moderncvstyle{classic}

\moderncvcolor{orange}



\usepackage{amsmath}

\usepackage{amsfonts}

\usepackage{amssymb}

\usepackage{graphicx}

\setcounter{MaxMatrixCols}{30}


% alcuni pacchetti standard

%\usepackage[italian]{babel} % solo se si scrive in italiano

\usepackage[utf8]{inputenx}

\usepackage[left=2cm,right=2cm,top=1.8cm,bottom=2.2cm]{geometry}


% questa riga allarga la colonna di sinistra

\setlength{\hintscolumnwidth}{3.7cm}


% personal data

\firstname{Christian}

\familyname{El Emam}

\title{Curriculum Vitae}

\address{University of Luxembourg \\ Mathematics Department, Maison du Nombre\\}{6, Avenue de la Fonte}{L-4364, Esch-Sur-Alzette, Luxembourg}

\phone{(+352) 46 66 44 9656}

%\mobile{}

\email{christian.elemam@uni.lu}

\homepage{math.uni.lu/elemam} 



\fancyfoot[L]{\footnotesize Christian El Emam's Curriculum}

\fancyfoot[R]{\footnotesize pag.~\arabic{page}}


\newcommand{\rr}{\mathbb{R}}

\renewcommand{\ss}{\mathbb{S}}


\newcommand*{\cventrym}[6]{%
	\cvline{#1}{%
		{#2}%
		\ifx#3\else{{\slshape#3}}\fi%
		\ifx#4\else{ #4}\fi%
		\ifx#5\else{ #5}\fi%
		\ifx#6\else{\newline{}\begin{minipage}[t]{\linewidth}\small#6\end{minipage}}\fi
}}%

\newcommand*{\cventrynovirg}[6]{%
	\cvline{#1}{%
		{\bfseries#2}%
		\ifx#3\else{{\slshape#3}}\fi%
		\ifx#4\else{ #4}\fi%
		\ifx#5\else{ #5}\fi%
		\ifx#6\else{\newline{}\begin{minipage}[t]{\linewidth}\small#6\end{minipage}}\fi
}}%





\begin{document}


\maketitle


\section{Personal Information}

\cvline{First name}{Christian}

\cvline{Last name}{El Emam}

\cvline{Date and place of birth}{06/11/1993 in Sesto San Giovanni (Italy)}

\cvline{Nationality}{Italian}



\section{Current position}

\cventry{January 2021 - now}
{Postdoctoral researcher}
{University of Luxembourg}
{}
{}
{Supervisor: Prof. Jean-Marc Schlenker}






\section{Education}


%\cventry{da ottobre 2016}{Dottorato di ricerca in Matematica}{Università degli Studi di Pavia}{}{}{Curriculum: Analisi Matematica}

\cventry{Oct. 2020 - Dec. 2020}
{Research and development specialist}
{University of Luxembourg}
{}
{}
{Supervisor: Prof. Jean-Marc Schlenker}


\cventry{December 2020}
{PhD in Mathematics}
{University of Pavia}
{}
{}
{Thesis title: \emph{Immersions of surfaces into $SL(2,\mathbb C)$ and into the space of geodesics of Hyperbolic space.}\\
	Advisor: Prof. Francesco Bonsante. \medskip
}



\cventry{October 2017}
{M.Sc. in Mathematics cum laude}
{University of Milan}
{}
{}
{Thesis title: \emph{Fenchel-Nielsen coordinates for Teichm\"uller space.}\\
Advisor: Prof. Francesco Bonsante. Co-Advisor: Prof. Gilberto Bini.\medskip
}

\cventry{Sept. 2016 - Feb. 2017}
{Erasmus stay} 
{at Universidad Complutense of Madrid, Master program "Matematicas Avanzadas"}
{}
{}
{}

\cventry{September 2015}
{Bc. in Mathematics cum laude}
{University of Milan}
{}
{}
{}


\cventry{July 2010}
{Scientific high school diploma}
{Liceo Scientifico "Giulio Casiraghi", Cinisello Balsamo (Milan) }
{}
{}
{\medskip}






\section{Research Interests}
\cvline{}{Geometric structures on surfaces, hyperbolic geometry, immersions of hypersurfaces in space forms, transition \nobreak{geometry}, higher Teichmüller theory.}




\section{Papers}
\cventrym{In progress} {(with N. Sagman) "On a Bers theorem for the quasi-Hitchin space of $PSL(3, \mathbb C)$"}{}{}{}{}

\cventrym{In progress} {(with F. Mazzoli, A. Seppi, A. Tamburelli) "A null-hyperKähler structure on the space of minimal surfaces in the Half-Pipe 3-space"}{}{}{}{}

\cventrym{2023}{ "A metric uniformization model for the Quasi-Fuchsian space",}{ submitted, arXiv:2307.07388.}{[$45$ pages]}{}{}

\cventrym{2022}{(with A. Seppi) "Rigidity of minimal Lagrangian diffeomorphisms between spherical cone surfaces",}{ Journal de l'École polytechnique — Mathématiques .}{[$20$ pages]}{}{}

\cventrym{2022}{(with A. Seppi) "On the Gauss map of equivariant immersions in hyperbolic space", } {Journal of Topology, 15: 238-301. https://doi.org/10.1112/topo.12225.} {}{[$64$ pages]}{}

\cventrym{2021}{(with F. Bonsante) "On immersions of surfaces in $SL(2,\mathbb{C})$ and geometric consequences", }{International Mathematics Research Notices, 2021; rnab189, https://doi.org/10.1093/imrn/rnab189. }{[$62$ pages]}{}{}

\cventrym{2020}
{"On $PSL(2,\mathbb C)$ and on the space of geodesics of $\mathbb H^3$ as Riemannian holomorphic manifolds",} 
{ Actes du séminaire Théorie Spectrale et Géométrie of Institut Fourier, 
	Volume 35 (2017-2019), p. 9-21.}{[$12$ pages]}{}{}

\subsection{PhD Thesis}
\cventrym{2020}
{PhD Thesis: "Immersions of surfaces into $SL(2,\mathbb C)$ and into the space of geodesics of Hyperbolic space".} 
{}{}{}{}





\section{Given and upcoming talks}


\cvline{05/2023}{\emph{"A metric uniformization of quasi-Fuchsian space",} Luxembourg.}

\cvline{02/2023}{\emph{"The holomorphic extension of the Weil-Petersson metric to the quasi-Fuchsian space",} Nice.}

\cvline{07/2022}{\emph{"Minimal Lagrangian maps between closed spherical cone surfaces are isometries",} AMS-EMS-SMF 2022 congress, Grenoble.}

\cvline{06/2022}{\emph{"Le mappe minimali lagrangiane tra superfici sferiche chiuse sono isometrie.",} Pavia.}

\cvline{30/03/2021}{\emph{"Immersions of surfaces into SL(2,C) as an approach to transition geometry",} Florida State University [telematic].}


\cvline{9/02/2021}{\emph{"Families of equivariant immersions in $\mathbb{H}^3$ with holomorphic holonomy",} Pangolin seminar
[telematic:
\href{https://www.youtube.com/watch?v=AjOhplLR_Rs}{\underline{link to the video}}].
}

\cvline{05/06/2020}{\emph{"Che forma ha un mondo perfetto? Un approccio alle varietà Riemanniane con molte simmetrie"}, Se mi narri di Matematica, University of Pavia [telematic:
\href{http://euler.unipv.it/seminaridott/4_El_Emam.html}{\underline{link to the video}}].}

\cvline{21/10/2019}{\emph{"On immersions of surfaces into $PSL(2,\mathbb C)$ and on a tool for constructing holomorphic maps into its character variety"}, Geometry and Topology seminar, University of Luxembourg.}

\cvline{18/04/2019}{\emph{"On  immersions of surfaces into the space of geodesics of $\mathbb{H}^3$: a link between complex-valued metrics and projective structures"}, Séminarie de théorie spectral et géométrie, Institut Fourier, Grenoble.}

\cvline{26/11/2018}{\emph{"On immersions of surfaces in $PSL(2,\mathbb{C})$ and on orthogonal structures on the complexified tangent bundle of surfaces"}, Geometry and Topology seminar, University of Luxembourg.}




\section{Events and seminar organization}
\cventrynovirg{28/08-01/09/2023}{Co-organizer of the summer school}{ "\href{https://sites.google.com/view/frejus-2023/home}{\underline{Metrics on higher Teichmüller spaces}}"}{in Fréjus.}
{}{\medskip}

\cventrynovirg{24-26/01/2022}{Co-organizer of the conference}{ "\href{https://math.uni.lu/geometry/winter2022/index.html}{\underline{Geometry Winter Workshop in Luxembourg}}"}{at the University of Luxembourg,}
{}{\medskip}


\cventrynovirg{2021-now}{Co-organizer}{ of the "Geometry and Topology seminar"}{at the Mathematics Department of the University of Luxembourg.}{}{\medskip}

\cventrynovirg{12/05/2020}{Co-organizer of the event }{"International Women in Mathematics Day"}{at the Mathematics Department of the University of Pavia.}
{}{\medskip}

\cventrynovirg{2020}{Co-organizer and co-founder of the PhD seminar}{ "Se mi narri di Matematica"}{at the Mathematics Department of the University of Pavia.}{}{\medskip}




\section{Thesis supervision}
\cventrynovirg{2022} {Bachelor final seminar supervision:}{ "The hairy ball theorem and Brouwer fixed-point theorem",} {Aleksandra Frania.}
{}
{}
{}

\cventrynovirg{2021} {Bachelor thesis:} { "Borsuk-Ulam theorem and applications",} {Yanis Bosch.}
{}
{}



\section{Research visits}


\cvline{April-May 2019} {Institut Fourier, Grenoble.}{}{}{}

\cvline{October 2020} {Mathematics Department, University of Luxembourg.}{}{}{}













\iffalse


\section{Student carreer} 

\cvline{2017}
{Ranking first place in the admission to the Joint PhD Program in Mathematics \emph{"Milano Bicocca-Pavia-INdAM"}}
{}
{}
{}
{}

\cvline{2015}
{Scholarship "Incentivi Lauree scientifiche" by University of Milan
	for being the second-year student with the highest average mark.}
{}
{}
{}

\cvline{2014}
{Scholarship for merit by University of Milan.}
{}
{}
{}
{}

\cvline{2014}
{Scholarship "Incentivi Lauree scientifiche" by University of Milan
	for being the second-year student with the highest average mark.
}
{}
{}
{}




\fi





\section{Teaching activities}
\cventry{A. Y. 2023-2024}{University of Luxembourg}{Exercise class: "Analysis 1"}{Bc. in Physics}{}{}

\cventry{A. Y. 2022-2023}{University of Luxembourg}{Exercise class: "Analysis 1"}{Bc. in Physics}{}{}

\cventry{A. Y. 2021-2022}{University of Luxembourg}{Main course: "Numerical Analysis for Physicists"}{Bc. in Physics}{}{}

\cventry{A. Y. 2021-2022}{University of Luxembourg}{Exercise class: "Partial differential equations 1"}{M.Sc. in Mathematics}{}{}

\cventry{A. Y. 2020-2021}{University of Luxembourg}{Exercise class: "Analysis 2"}{Bc. in Mathematics}{}{}

\cventry{A. Y. 2019-2020}{University of Pavia}{Exercise class: "Analysis 1"}{Bc. in Engineering}{}{}

\cventry{A. Y. 2018-2019}{University of Pavia}{Lectures of "Preparation for first-year students in Mathematics"}{Bc. in Mathematics}{}{}

\cventry{A. Y. 2017-2018}{University of Milan}{Exercise class: "Geometry 4"}{Bc. in Mathematics}{}{}

\cventry{A. Y. 2016-2017}{University of Milan}{Exercise class: "Geometry 1"}{Bc. in Mathematics}{}{}



\iffalse

\section{Attended conferences}
\cvline{February 2020} {\emph {Jíbiri + Málaga \& Topology Meeting}, University of Málaga.}

\cvline{June 2019} {\emph{Geometrie et dynamique de A a Z}, University of Avignon.}

\cvline{February 2019}{\emph{Geometric Analysis meets Geometric Topology}, University of Heidelberg.}

\cvline{January 2019} {\emph{Winter School on Geometric Structures in Nice} and \emph{Conference on Geometric Structures in Nice}, University of Nice.}

\cvline{August 2018}{\emph{Summer School on Teichmüller Theory and its Connections to Geometry, Topology and Dynamics} and \emph{Workshop on Geometry of Teichmüller Space}, Fields Institute, Toronto.}

\cvline{July 2018}{\emph{Representation varieties and geometric structures in low dimensions}, University of Warwick.}

\cvline{June 2018}{\emph{Pseudo-Riemannian geometry and Anosov representations}, University of Luxembourg.}

\cvline{April 2018}{\emph{Geometry of Teichm\"uller space and mapping class groups}, University of Warwick.}

\cvline{November 2017}{\emph{Geometry, groups and dynamics}, ICTS, Bangalore.}



\fi





















\section{Language Skills}

\cvline{Italian}{native speaker}

\cvline{English}{advanced}

\cvline{Spanish}{very good (aunque estoy un poco oxidado)}

\cvline{French}{good}







\vspace{\fill} 

%Luxembourg, \today


\end{document}